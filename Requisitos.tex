\documentclass{article}
\usepackage{graphicx} % Required for inserting images
\usepackage{subcaption} % Para usar subfiguras, etc.
\usepackage[spanish]{babel} % Para que la fecha sea en español
\usepackage[colorlinks=true, linkcolor=blue, urlcolor=blue]{hyperref} % Para crear hipervínculos
\usepackage{wrapfig} % Texto al lado de una figura
\usepackage{multirow} % Permite juntar dos celas verticales de una tala
\usepackage{colortbl} % Permite colorear tablas
\usepackage[table]{xcolor} % Permite colorear una tabla de forma alterna
\usepackage{lipsum} % Permite usar /lipsum para crear un texto largo
\usepackage[left=25mm, right=25mm, top=35mm, bottom=30mm, headheight=35mm]{geometry} % Permite definir los márgenes
\usepackage{fancyhdr} % Permite usar encabezados

% Definición de colores
\definecolor{celeste}{RGB}{0, 255, 255}
\definecolor{gris}{RGB}{222, 222, 222} % Idóneo para alternar en las tablas

\graphicspath{{./images/}}

% Encabezado
\pagestyle{fancy}
\lhead{\includegraphics[width=0.12\textwidth]{logo.jpg}}
\chead{TFG Votaciones}
\renewcommand{\headrulewidth}{2pt}

\title{Votaciones}
\author{El trío calavera}

\begin{document}

\maketitle

\tableofcontents
\newpage

\section{Requisitos funcionales}
    \subsection{Proceso general de votación}
    Nuestra página estará orientada a las elecciones de delegado para cada curso y grupo, y a continuación se detallan los pasos a seguir:
    \begin{enumerate}
        \item La página principal del programa contendrá un formulario de login o registro para el usuario. Este contará con
        un segundo factor de autenticación para proporcionar una mayor autenticidad.
        \item Una vez estás logeado con permisos de creación de elecciones (por ejemplo, los profesores), te saldrá la pestaña de creación
        de elecciones. Una vez aquí, tendrás que elegir:
        \begin{itemize}
            \item Un título (pe: Elección de delegado 1ºA).
            \item La cantidad de participantes (un número entre 1 y 150).
            \item Fecha de inicio de la votación.
            \item Fecha final de la votación.
            \item Los correos electrónicos de todos aquellos que tengan derecho a votar.
            \item Una foto (opcional).
        \end{itemize}
        La página creará las elecciones y le asignará un código que deberás compartir con los alumnos para que estos puedan acceder ella.
        \item Los alumnos, tras logearse con permisos de usuario base, podrán buscar la elección con el código proporcionado por el
        profesor y presentarse como candidato, consultar las distintas candidaturas de otros alumnos, y llevar a cabo su voto.
        \begin{itemize}
            \item Presentarse como candidato: para ello tendrán que presionar el botón correspondiente y rellenar un formulario que incluye:
            \begin{itemize}
                \item Foto del participante.
                \item Eslogan de su candidatura.
                \item Texto en el que incluya sus objetivos, motivaciones, etcétera.
                \item Vídeo de presentación (opcional).
            \end{itemize}
            \item Consultar candidaturas: haciendo click en el botón correspondiente les aparecerá una lista con todos los nombres y
            fotos de las candidaturas, y presionando en ellas las mostrará en detalle.
            \item Votar: el alumno podrá, si está dentro de la fecha, elegir a su candidato favorito y votarle. Posteriormente tendrá
            la posibilidad de descargar un comprobante de su voto.
        \end{itemize}
        \item Finalmente todos los alumnos y profesor podrán consultar el ganador de las elecciones una vez acabado el plazo de votación.
    \end{enumerate}
    \subsection{Otras características de la aplicación}
\newpage

\section{Requisitos no funcionales}

\newpage

\end{document}