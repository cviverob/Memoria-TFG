\documentclass{article}
\usepackage{graphicx} % Required for inserting images
\usepackage{subcaption} % Para usar subfiguras, etc.
\usepackage[spanish]{babel} % Para que la fecha sea en español
\usepackage[colorlinks=true, linkcolor=blue, urlcolor=blue]{hyperref} % Para crear hipervínculos
\usepackage{wrapfig} % Texto al lado de una figura
\usepackage{multirow} % Permite juntar dos celas verticales de una tala
\usepackage{colortbl} % Permite colorear tablas
\usepackage[table]{xcolor} % Permite colorear una tabla de forma alterna
\usepackage{lipsum} % Permite usar /lipsum para crear un texto largo
\usepackage[left=25mm, right=25mm, top=35mm, bottom=30mm, headheight=35mm]{geometry} % Permite definir los márgenes
\usepackage{fancyhdr} % Permite usar encabezados

% Definición de colores
\definecolor{celeste}{RGB}{0, 255, 255}
\definecolor{gris}{RGB}{222, 222, 222} % Idóneo para alternar en las tablas

\graphicspath{{./images/}}

% Encabezado
\pagestyle{fancy}
\lhead{\includegraphics[width=0.12\textwidth]{logo.jpg}}
\chead{TFG Votaciones}
\renewcommand{\headrulewidth}{2pt}

\title{Votaciones}
\author{El trío calavera}

\begin{document}

\maketitle

\tableofcontents
\newpage

\section{Requisitos del programa}
\subsection{Seguridad ante ataques}
\begin{itemize}
    \item Código cerrado para dificultar el proceso a los atacantes, de tal manera que 
    nosotros controlamos todo el sistema de votación.
    \item Sistema de autenticación criptográfica para impedir múltiples votos por parte 
    de una persona y que un usuario externo pueda votar.
    \item Auditorías aleatorias para detectar fallos o modificaciones del software.
    \item Verificación de que un cliente no acceda habiendo cargado un script malicioso.
    \item Robustez frente a la denegación de servicio.
    \item Uso de criptografía y funciones resumen para proteger un voto.
\end{itemize}


\end{document}